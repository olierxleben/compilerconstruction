\section{Kommandozeilenwerkzeug}
Um Optimierungen durchzuführen wurde im Rahmen der Hausarbeit ein Kommandozeilenwerkeug erstellt, welches in den folgenden Teilabschnitten vorgestellt wird.

\subsection{Makefile und Start des Programms}
Zur Kompilierung und Bereitstellung wurde ein Makefile erstellt (Details siehe Listing \ref{makefile_listing}). Mit dem Makefile besteht ebenfalls die Möglichkeit Einzelteile des Projekts zu erstellen. Zum Beispiel kann mittels \textbf{make parser} nur der Parser erstellt werden.
% TODO: verfassen
\begin{lstlisting}[label=makefile_listing,language=C, caption=Auszug aus dem Makefile des Projekt]

all: app
	
app: lex.yy.c test.tab.c test.tab.h grammar/css_types.h
	cc grammar/test.tab.c grammar/lex.yy.c grammar/css_types.c main.c cli_parse.c css_merge.c guiCSS.c optimizer.c output.c -lncurses -o optimCSS
	
parser: lex.yy.c test.tab.c test.tab.h grammar/css_types.h
	cc grammar/test.tab.c grammar/css_types.c grammar/lex.yy.c -o parser

\end{lstlisting}

\subsection{Datenstrukturen}

\begin{lstlisting}[label=datastructures_listing,language=C, caption=Datenstrukturen]

\end{lstlisting}

\subsection{Einlesen der HTML- und CSS-Dateien}
% TODO: verfassen

\subsection{Zusammenführen der CSS-Dateien}
% TODO: verfassen

\subsection{Teilergebnis}
% TODO: verfassen