\section{Kommandozeilenwerkzeug}
Um Optimierungen durchzuführen wurde im Rahmen der Hausarbeit ein Kommandozeilenwerkeug erstellt, welches in den folgenden Teilabschnitten vorgestellt wird.

\subsection{Makefile und Start des Programms}
Zur Kompilierung und Bereitstellung wurde ein Makefile erstellt (Details siehe Listing \ref{makefile_listing}). Mit dem Makefile besteht ebenfalls die Möglichkeit Einzelteile des Projekts zu erstellen. Zum Beispiel kann mittels \textbf{make parser} nur der Parser erstellt werden.
% TODO: verfassen
\begin{lstlisting}[label=makefile_listing,language=C, caption=Makefile]
#include grammar/makefile.mk

all: app
	
app: lex.yy.c test.tab.c test.tab.h grammar/css_types.h
	cc grammar/test.tab.c grammar/lex.yy.c grammar/css_types.c main.c cli_parse.c css_merge.c guiCSS.c optimizer.c output.c -lncurses -o optimCSS
	
parser: lex.yy.c test.tab.c test.tab.h grammar/css_types.h
	cc grammar/test.tab.c grammar/css_types.c grammar/lex.yy.c -o parser

test.tab.c test.tab.h: grammar/test.y
	bison -d grammar/test.y -o grammar/test.tab.c
            
lex.yy.c: grammar/test.l test.tab.h
	flex -o grammar/lex.yy.c grammar/test.l

clean: 
	rm -f lex.yy.c grammar/test.tab.h grammar/test.tab.c grammar/parser optimCSS
\end{lstlisting}

\subsection{Verwendete Bibliotheken}

\subsubsection{Curses/NCurses}

Zur Ausgabe der Kommandozeile wurde die Programmierbibliothek \textbf{curses}, bzw. \textbf{ncurses} verwendet. Curses wird zum Screen Handling und dessen Optimierung verwendet. Es liegt allen gängigen Linux-Distributionen in den Standartpaketquellen vor. Unter Mac OS X wird die curses-Bibliothek mit den Developer Tools von Xcode mit installiert. Unter Windows kann Cygwin oder MingGW genutzt werden um die Bibliothek zu erhalten. Weitere Informationen können den \textit{man pages}\footnote{man page (unix): \textbf{man ncurses} auf der Kommandozeile} entnommen werden. 

Wie im Listing \ref{makefile_listing} zu sehen ist, wird die Anwendung (\textit{app}) mittels \textbf{-lncurses} kompiliert. 

\subsubsection{Gumbo - HTML5 Parser}
Für Optimierungen werden Inhalte der zugrundeliegenden HTML-Datei benötigt. So ist es beispielsweise notwendig alle HTML-Tags, Klassen und IDs die in den CSS-Dateien definiert wurden mit den vorhandenen Tags der HTML-Datei zu vergleichen. Unbenutzte Definitionen sind im CSS somit unnötig und können entfernt werden.

Zum Parsen der HTML-Datei wird der Parser \textit{Gumbo} verwendet. Gumbo wurde im August 2013 von Google als Open Source - Projekt freigegeben. Die Bibliothek selbst ist in C geschrieben und hält sich an den HTML-Parsing-Algorithmus\footnote{(X)HTML-Parsing-Algorithmus:http://www.whatwg.org/specs/web-apps/current-work/multipage/parsing.html}. Die Quellen\footnote{https://github.com/google/gumbo-parser/} liegen auf Github. Das Listing \ref{gumbo_install} zeigt die notwendigen Instruktionen um Gumbo nach erfolgreichen herunterladen (auf einem unixoiden System) zu installieren.

\begin{lstlisting}[label=gumbo_install,language=bash, caption=Installation der gumbo Bibliothek]
$ ./autogen.sh
$ ./configure
$ make
$ (sudo) make install
\end{lstlisting}

Die detailierte Verwendung wird im Abschnitt \ref{unused_css_delete} vorgestellt. Das Listing \ref{makefile_listing}, Zeile 6 verwendet \textbf{lgumbo} um den Compiler mizuteilen, dass die Anwendung damit kompiliert werden soll. 

\subsection{Hauptroutine}
% TODO: verfassen

\subsection{Einlesen der HTML- und CSS-Dateien}
% TODO: verfassen

\subsection{Zusammenführen der CSS-Dateien}
% TODO: verfassen

\subsection{Teilergebnis}
% TODO: verfassen