\section{Kommandozeilenwerkzeug}
Um Optimierungen durchzuführen wurde im Rahmen der Hausarbeit ein Kommandozeilenwerkeug erstellt.

\subsection{Curses/NCurses}

Zur Ausgabe der Kommandozeile wurde die Programmierbibliothek \textbf{curses}, bzw. \textbf{ncurses} verwendet. Curses wird zum Screen Handling und dessen Optimierung verwendet. Es liegt allen gängigen Linux-Distributionen in den Standartpaketquellen vor. Unter Mac OS X wird die curses-Bibliothek mit den Developer Tools von Xcode mit installiert. Unter Windows kann Cygwin oder MingGW genutzt werden um die Bibliothek zu erhalten. Weitere Informationen können den Man pages entnommen werden.
% TODO: Beispiel ?!

\subsection{Einlesen der CSS-Dateien}
% TODO: verfassen
\subsection{Zusammenführen der CSS-Dateien}
% TODO: verfassen
\subsection{Terilergebnis}
% TODO: verfassen