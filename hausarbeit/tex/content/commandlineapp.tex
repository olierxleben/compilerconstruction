\section{Kommandozeilenwerkzeug}
Um Optimierungen durchzuführen wurde im Rahmen der Hausarbeit ein Kommandozeilenwerkeug erstellt, welches in den folgenden Teilabschnitten vorgestellt wird.

\subsection{Makefile und Start des Programms}
Zur Kompilierung und Bereitstellung wurde ein Makefile erstellt (Details siehe Listing \ref{makefile_listing}). Mit dem Makefile besteht ebenfalls die Möglichkeit Einzelteile des Projekts zu erstellen. Zum Beispiel kann mittels \textbf{make parser} nur der Parser erstellt werden.
% TODO: verfassen
\begin{lstlisting}[label=makefile_listing,language=C, caption=Makefile]
#include grammar/makefile.mk

all: app
	
app: lex.yy.c test.tab.c test.tab.h grammar/css_types.h
	cc grammar/test.tab.c grammar/lex.yy.c grammar/css_types.c main.c cli_parse.c css_merge.c guiCSS.c optimizer.c output.c -lncurses -o optimCSS
	
parser: lex.yy.c test.tab.c test.tab.h grammar/css_types.h
	cc grammar/test.tab.c grammar/css_types.c grammar/lex.yy.c -o parser

test.tab.c test.tab.h: grammar/test.y
	bison -d grammar/test.y -o grammar/test.tab.c
            
lex.yy.c: grammar/test.l test.tab.h
	flex -o grammar/lex.yy.c grammar/test.l

clean: 
	rm -f lex.yy.c grammar/test.tab.h grammar/test.tab.c grammar/parser optimCSS
\end{lstlisting}

Der Start des Programms geschieht über die Konsole unter Angabe von Parametern. Zur Verfügung stehen die folgenden Parameter:

\begin{description}[label=description_program_start]
   \item[-f dateiname] Angabe der Quelldatei. Dies muss eine HTML-Datei sein. Aus der angegebenen HTML-Datei werden die genutzten CSS-Dateien extrahiert und zusammengeführt für die Optimierung.
   \item[-m] Minifizierte Ausgabe.
   \item[-s] Strukturierte Ausgabe.
\end{description}

Zum Start des Programms ist der Parameter -f zwingend nötig um dem Optimierer CSS-Dateien zur Verfügung zu stellen. Ein Beispiel ist in Listing \ref{example_program_start} dargestetllt.

\begin{lstlisting}[label=example_program_start,language=Bash, caption=Programmstart]

./optimCSS -f index.html

\end{lstlisting}

\subsection{Datenstrukturen}

Zum Einlesen der Kommandozeilen parameter und der HTML- bzw. CSS-Dateien wurden Datenstrukturen definiert. Diese sind in Listing \ref{datastructures_input_listing} dargestellt.
Mit Hilfe der Strukturen ist es möglich den Ausgabetyp, Eingabetyp, Eingabedatei sowie eine temporäre gemergte CSS-Datei zu halten.

\begin{lstlisting}[label=datastructures_input_listing,language=C, caption=Datenstrukturen]

enum output_type {STRUCTURED, MINIFIED};  // -s, -m
enum input_type {FILE_INPUT, PATH_INPUT}; // -f, -p

struct input_data {
	int output_type;
	int input_type;
	char* src;
};

struct css_data {
	char* src_html;
	char* merged_css;
};

\end{lstlisting}


\subsection{Grundaufbau des Kommandozeilenwerkzeugs}
Das Kommandozeilenwerkzeug lässt sich in fünf Komponenten unterteilen. In der ersten Komponente werden die Kommandozeilenparameter eingelesen und für die spätere Nutzung in einer Struktur abgelegt.
Die folgende Komponente ermittelt aus der übergebenen HTML-Datei alle genutzten CSS-Dateien und merged diese zu einer temporären CSS-Datei, welche später zur Optimierung verwendet wird.
Nachdem nun eine einzelne CSS-Datei zum Optimieren bereit steht wird diese zunächst mittel eines durch Flex und Bison erzeugten Parsers in eine Baumstruktur überführt und so für die eigentiche Optimierung aufbereitet.
Die hierzu genutzten Flex- und Bison-Quellen sind in den Listings \ref{css_lex} und \ref{css_bison} dargestellt. 
Auf dem nun vorliegenden CSS-Baum werden nachfolgend einige Optimierungen durchgeführt, welche in Kapitel \ref{chap_optimization} detaillierter beschrieben werden.
Abschließend wird der optimierte CSS-Baum in Textform als CSS-Datei ausgegeben. Die Ausgabe kann sowohl strukturiert (gut lesbar) als auch minifiziert erfolgen. Dies wird mittels Kommandozeilenparameter eingestellt.



