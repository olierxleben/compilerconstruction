\section{Einführung in CSS}
\subsection{Cascading Stylesheets}

Cascading Stylesheets gilt aktuell als Standard-Stylesheetsprache für Websites. Es findet aber auch Verwendung in anderen Umgebungen, wie zum Beispiel JavaFX. Mit CSS ist es möglich Ausgabemedien unterschiedliche Darstellung vorzugeben. Beispielsweise soll einen Hyperlink\footnote{Hyperlink: http://www.w3schools.com/html/html\_links.asp} beim Druck einer Seite in einer anderen Farbe dargestellt werden oder die Elemente einer Bildergalerie sollen auf Geräten mit kleinerer Auflösung (wie etwa Tablets oder Smartphones) gelistet werden und bei einem Display mit FullHD-Auflösung können Bilder in einem Gitternetz dargestellt werden. 

Verschiedene spezielle CSS-Eigenschaften können für die unterschiedlichen Einsatzgebiete von CSS existieren. Die vorliegende Ausarbeitung bezieht sich auf CSS im Web-Stack, bzw. untersucht nur CSS für Webseiten. 

\subsection{Revisionen und Probleme}
Die Entwicklung von CSS wurde 1993 begonnen und hat nach mehreren Versionen 1995 den Stand der ersten Version erreicht. 1998 wurde die Version zwei vorgestellt. Diese ist aktuell und wurde bis heute allerdings von keinem der Mainstream-Browser vollständig implementiert. Einige Browser setzen große Teile des CSS2, bzw. CSS2.1-Standards korrekt um. Es kann aber zu Problemen bei unterschiedlichen Browsern führen. 2011 wurde die Recommandation für CSS 2.1 veröffentlicht. 

Die aktuelle Entwicklung für den dritten Standard von CSS begann bereits 2000. Zu den Neuerungen in dieser Revision gehören vor allem der modulare Aufbau, womit CSS-Eigenschaften nun schrittweise entwickelt werden können. Fast alle Browser implementieren Kernfunktionen des dritten CSS-Standards, die allerdings variieren können. 

\subsection{Allgemeiner Aufbau}

Damit eine Optimierung von bestehenden CSS-Quelltext durchgeführt werden kann, muss zuvor analysiert werden wie ein Browser CSS verarbeitet und wie Kaskadierung im Browser genutzt wird um Style-Informationen zu verknüpfen.

Eine CSS-Datei besteht aus einer Menge von Anweisungen, oder Regeln, die das Layout einzelner Elemente (mit bestimmter ID), Klassen von Elementen oder allgemein gültigen Elementen bestimmen. Regeln deren anzuwendenen Eigenschaften gleich sind, können mittels Kommata getrennt werden.

Der allgemeine Aufbau einer CSS-Regel wird im Listing \ref{css_basics} veranschaulicht.   

\begin{lstlisting}[label=css_basics,language=bash, caption=Aufbau einer CSS-Regel]
a {
    background:#000;
    color: #fff;
}  
#special {
    border: 1px solid red;
}

.left, top.myClass {
    border: 1px dotted green;
}
\end{lstlisting}

Im Listing \ref{css_basics} sind drei Regeln gegeben. Die erste Regel wird für ein HTML-Element angewendet. Jedes a-Element (Hyperlinks im Browser) werden mit schwarzen Hintergrund angezeigt und der textliche Inhalt des Elements wird weiß dargstellt. Bei dem zweiten Element wird eine Regel auf ein Element mit der ID \textbf{special} angwendet. Dies erhält einen durchgehenden roten Rahmen um den Inhalt des Elements. Die dritte Regel legt eine gepunktete grüne Linie um den Inhalt der Elemente mit der Klasse \textbf{left}, sowie der HTML-Elemente top mit der zugewiesen Klasse \textbf{myClass}.  

\subsection{Grammatik}

Die zugrundeliegende Grammatik stammt von der Seite des W3C (Details, siehe: \cite{w3c_css_grammar}). Dies entspricht im Wesentlichen der Grammatik für die CSS Version 2.1. Das Listing \ref{w3c_grammar} zeigt einen Auszug der Grammatik des W3C. 

\begin{lstlisting}[label=w3c_grammar,language=C, caption=Grammatik des W3C]

 /* ... */
ruleset
  : selector [ ',' S* selector ]*
    '{' S* declaration? [ ';' S* declaration? ]* '}' S*
  ;
selector
  : simple_selector [ combinator selector | S+ [ combinator? selector ]? ]?
  ;
simple_selector
  : element_name [ HASH | class | attrib | pseudo ]*
  | [ HASH | class | attrib | pseudo ]+
  ;
class
  : '.' IDENT
  ;
element_name
  : IDENT | '*'
  ;
attrib
  : '[' S* IDENT S* [ [ '=' | INCLUDES | DASHMATCH ] S*
    [ IDENT | STRING ] S* ]? ']'
  ;
pseudo
  : ':' [ IDENT | FUNCTION S* [IDENT S*]? ')' ]
  ;
declaration
  : property ':' S* expr prio?
  ;
prio
  : IMPORTANT_SYM S*
  ;
expr
  : term [ operator? term ]*
  ;

/* ... */

\end{lstlisting}

Für die Optimierung einer CSS-Datei ist diese Grammatik allerdings ungeeignet. Zum einen unterstützt die Grammatik nicht offiziell CSS3-Elemente, wenn auch Teile davon, und zum anderen ist die Aufgabe des Optimierers nicht die Darstellung oder die Interpretation der CSS-Quelldateien, sondern die Verbesserung der definierten Elemente nach definierten Regeln. Dafür wurde eine eigene Grammatik entwickelt. Diese wird im Abschnitt \ref{tree_generation_lex} vorgestellt. 


