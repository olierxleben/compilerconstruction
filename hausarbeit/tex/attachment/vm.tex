
\section{Virtuelle Maschine}
Zur Messung der Testdaten wird eine virtuelle Maschine erstellt. Diese liegt dem Datenträger als VirtualBox-Maschine dieser Hausarbeit bei. 

\subsection{Betriebssystem}
Als Betriebssystem dient ein aktuelles Debian Linux mit einem virtuellen Kern, 256 MB Arbeitsspeicher und 8 GB Massenspeicher. Diese Hardwarespezifikation entspricht im wesentlichen eingebetteten Systemen oder Shared Webhosting. 

\subsection{Benutzer- und Logindaten}

Für die VM wurde folgende Benutzer mit Passwort angelegt:
\begin{itemize}
    \item{cbh:compiler (Standardbenutzer)}
    \item{root:r00t (Superuser)}
\end{itemize}

\subsection{Dienste und Tools}

Auf der virtuellen Maschine sind folgende Dienste und Werkzeuge installiert: 

\begin{itemize}
    \item{FTP}
    \item{Ruby}
    \item{SSH}
    \item{Git}
\end{itemize}

Dem FTP- und SSH-Server ist der Standardbenutzer zugewiesen, sodass Dateien auf, bzw. von dem Server geladen werden können.
% TODO: Abschnitt beenden, Beispiel mit FTP und SSH/SCP  

\subsection{Ordnerstruktur}
Im Homeverzeichnes des Benutzers \textit{cbh} befindet sich das Git-Projekt \textit{compilerconstruction}.
% TODO: Beenden